%%%%%%%%%%%%%%%%%%%%%%%%%%%%%%%%%%%%%%%%%
% Beamer Presentation
% LaTeX Template
% Version 1.0 (10/11/12)
%
% This template has been downloaded from:
% http://www.LaTeXTemplates.com
%
% License:
% CC BY-NC-SA 3.0 (http://creativecommons.org/licenses/by-nc-sa/3.0/)
%
%%%%%%%%%%%%%%%%%%%%%%%%%%%%%%%%%%%%%%%%%

%----------------------------------------------------------------------------------------
%	PACKAGES AND THEMES
%----------------------------------------------------------------------------------------

\documentclass{beamer}

\mode<presentation> {

% The Beamer class comes with a number of default slide themes
% which change the colors and layouts of slides. Below this is a list
% of all the themes, uncomment each in turn to see what they look like.

%\usetheme{Antibes}
%\usetheme{Bergen}
%\usetheme{Berkeley}
%\usetheme{Berlin}
%\usetheme{Boadilla}
%\usetheme{CambridgeUS}
%\usetheme{Copenhagen}
%\usetheme{Darmstadt}
%\usetheme{Dresden}
%\usetheme{Frankfurt}
%\usetheme{Goettingen}
%\usetheme{Hannover}
%\usetheme{Ilmenau}
%\usetheme{JuanLesPins}
%\usetheme{Luebeck}
\usetheme{Madrid}
%\usetheme{Malmoe}
%\usetheme{Marburg}
%\usetheme{Montpellier}
%\usetheme{PaloAlto}
%\usetheme{Pittsburgh}
%\usetheme{Rochester}
%\usetheme{Singapore}
%\usetheme{Szeged}
%\usetheme{Warsaw}

% As well as themes, the Beamer class has a number of color themes
% for any slide theme. Uncomment each of these in turn to see how it
% changes the colors of your current slide theme.

%\usecolortheme{albatross}
%\usecolortheme{beaver}
%\usecolortheme{beetle}
%\usecolortheme{crane}
%\usecolortheme{dolphin}
%\usecolortheme{dove}
%\usecolortheme{fly}
%\usecolortheme{lily}
%\usecolortheme{orchid}
%\usecolortheme{rose}
%\usecolortheme{seagull}
%\usecolortheme{seahorse}
%\usecolortheme{whale}
%\usecolortheme{wolverine}

%\setbeamertemplate{footline} % To remove the footer line in all slides uncomment this line
%\setbeamertemplate{footline}[page number] % To replace the footer line in all slides with a simple slide count uncomment this line

%\setbeamertemplate{navigation symbols}{} % To remove the navigation symbols from the bottom of all slides uncomment this line
}

\usepackage{graphicx} % Allows including images
\usepackage{booktabs} % Allows the use of \toprule, \midrule and \bottomrule in tables

\usepackage[utf8]{inputenc}
\usepackage[english,russian]{babel}


% ----------------------------------------------------------------------------------------
%	TITLE PAGE
%----------------------------------------------------------------------------------------

\title{Создание кастомизированного эмбеддинга слов с помощью
  нейронных сетей} % The short title appears at the bottom of every slide, the full title is only on the title page

\author{Рустам Гилазтдинов\\
\textit{hsec@ro.ru}} % Your name

\date{\ 29 августа 2017 г. } % Date, can be changed to a custom date

\begin{document}

\begin{frame}
\titlepage % Print the title page as the first slide
\end{frame}

\begin{frame}
\frametitle{Обоснование подхода} % Table of contents slide, comment this block out to remove it
\tableofcontents % Throughout your presentation, if you choose to use \section{} and \subsection{} commands, these will automatically be printed on this slide as an overview of your presentation
\end{frame}

%----------------------------------------------------------------------------------------
%	PRESENTATION SLIDES
%----------------------------------------------------------------------------------------

%------------------------------------------------
\section{Обработка исходных данных} 

\begin{frame}
\frametitle{Обработка исходных данных}
Данные получены с ресурса Kaggle.
Необходимо построить свой кастомизированный эмбеддинг слов. Тут
сразу же возникает два варианта:
\begin{itemize}
\item {использовать готовую модель, такую как word2vec или GloVe, и дообучить ее
  на своих данных. В дальнейшем выяснилось, что такой подход хуже
  сказывается на метрике качества}
\item {натренировать свои эмбедденги, что в конечном итоге дало лучшее значение
    метрики, поэтому этот подход остался в финальном решении.}
\end{itemize}
  Будем решать задачу регрессии, так как оценки
- непрерывные величины. В целом, распределение оценок равномерное, поэтому можно
смело использовать mse, т.к. L2-loss должен сработать. Также существует подход
- разбить таргет на бины, но мы обойдемся простым случаем. Есть интересная
\href{https://arxiv.org/abs/1612.00775} {\textit {\color{red} статья}} на эту тему.
В описании к данным указано, что некоторые эссе имеют оценки от двух разных
групп рецензентов, в связи с этим я усреднил оценки по всем группам.
\end{frame}

%------------------------------------------------
\section{Описание архитектуры}

\begin{frame}
  \frametitle{Описание архитектуры}
Из специфики задачи было ясно, что нужно в первую очередь попробовать RNN, так
как данные текстовые. В совокупности с LSTM-слоем это послужило
baseline-решением. Следующей идеей стало следующее - да, у нас есть текстовые
данные, и RNN очень хорошо работают с последовательностями, но природа данных
такова, что, несомненно, длина эссе повлияла на конечную оценку, но помимо этого, существуют также внутренние
характеристики эссе, такие как связность, информативность и целостность текста и т.п. Поэтому было решено
использовать архитектуру, скомбинированную из RNN и CNN, что показывает себя как
state-of-the-art на многих датасетах. Таким образом, представление текста перед LSTM-слоем
является конкатенацией выходов нескольких сверток с разной шириной фильтра и ядра.
Вдохновением послужила статья
\href{https://arxiv.org/abs/1510.03820} {\textit{\color{red} ``A Sensitivity Analysis of Convolutional Neural Networks for Sentence Classification''}}.
\end{frame}

% ------------------------------------------------
\begin{frame}
  \frametitle{Описание архитектуры}
  \begin{figure}
    \centering
    \includegraphics[width=0.8\linewidth]{model.png}
    \caption{Архитектура сети}
  %\label{fig:model_nn}
\end{figure}
\end{frame}

% ------------------------------------------------

\section{Выбор фреймворка}
\begin{frame}
  \frametitle{Выбор фреймворка}
  \begin{figure}
    \centering
    \includegraphics[width=0.5\linewidth]{keras-tensorflow-logo.png}
  \end{figure}
Я использовал стандартный набор - keras и tensorflow
\begin{itemize}
\item functional api в керасе
\item tensorboard у tensorflow
\item относительно готов к продакшену в силу присутствия tf serving
\end{itemize}
Но еще есть PyTorch, который позволяет описывать граф вычислений не статично, а
императивно, что удобно в отладке и экспериментировании. Пока сыроват, но стоит
понаблюдать.
\end{frame}

%------------------------------------------------
\section{Результаты обучения}
\begin{frame}
  \frametitle{Результаты обучения}
  Так как решается задача линейной регрессии, я использую две самые простые и
  репрезентивные метрики - это \textbf{r2-score} и \textbf{коэффициент объясненной дисперсии}.
  Также, я дополнил их метрикой \textbf{quadratic weighted kappa}, которая использовалась
  в исходной задаче на платформе kaggle.
  \begin{table}
    \begin{tabular}{l l l}
      \toprule
      \textbf{Метрика} & \textbf{Значение}\\
      \midrule
      r2-score & 0.942 \\
      explained variance score & 0.944 \\
      kappa & 0.9688 \\
      \bottomrule
    \end{tabular}
    \caption{Результаты обучения}
  \end{table}
\end{frame}

%------------------------------------------------

\begin{frame}
  \frametitle{Результаты обучения}
\begin{block}{Пример 1}
  Some experts are concerned that people are spending to much time on computers,
  and I believe that they are correct. Most kinds would perfer to be on the
  computer chatting with friends on line or spending hours on facebook but what
  is the point of this when there is a whole to explore,  ... 
\end{block}
\textbf{Оценка преподователя} 11 & \textbf{Оценка моделью} 10.61144638
\begin{block}{Пример 2}
  Almost everyone is effected by computers in some sort of way. Some people think
  computers are bad because some people spend too much time on their computers.
  The thing is computers help us with our everyday life. If we want to...
\end{block}
\textbf{Оценка преподователя}  7 & \textbf{Оценка моделью} 7.3123
\end{frame}

%------------------------------------------------
\section{Параметры и их влияние}
\begin{frame}
  \frametitle{Параметры и их влияние}
Здесь хотелось бы отметить следующие моменты
\begin{itemize}
\item функция оптимизации - rmsprop лучше всего подходит для RNN
\item функция потерь - регрессия, поэтому стандартный mse
\item функция активации - использовал relu, лучший вариант
\item дропаут - важнейший параметр для регуляризации, но у нас данных не так
  много, поэтому небольшой. В изображениях, например, жалеть не нужно
\item размер батча - меньший батч не дает прироста метрик, а сеть учиться
  медленнее
\item размер фильтра и ядра в свертках - в целом, при добавлении сверточных
  слоев точность выросла, и добавление разных значений дало ощутимый прирост
\end{itemize}
\end{frame}

%------------------------------------------------
\section{Use cases}
\begin{frame}
  \frametitle{Use cases}
\begin{itemize}
\item кастомизированный эмбеддинг - отличное подспорье в domain-specific задачах
  обработки естественного языка 
\item эмбеддингами мы можем описать не только слова, а например, посещения
  пользователем набора сайтов, и создавать сегменты
\item эмбеддинги используются в биоинформатике - для векторизации геномов
\item используются в анализе музыки - классические произведения представлены нотами
\item используются в генерации последовательностей, причем, допустим, есть
  вектор с изображениями и их описанием, так вот, по интуиции,
  эмбеддинги будут способны помогать в классификации изображений, которых не
  было в обучающей выборке
\item в целом, эмбеддинги - универсальный 'репрезентатор' данных, то есть данные
  разной природы могут использовать одни и те же эмбеддинги
\end{itemize}
\end{frame}

%----------------------------------------------------------------------------------------

\end{document}